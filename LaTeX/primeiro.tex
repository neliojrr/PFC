\section{Introdu��o}

Criado por volta dos anos 50, o nome Compilador se refere ao processo 
de composi��o de um programa atrav�s da reuni�o de v�rias rotinas de bibliotecas. 
O processo de tradu��o (de uma linguagem fonte para uma linguagem objeto), 
considerado hoje a fun��o central de um compilador, era ent�o conhecido como 
programa��o autom�tica[Rangel, 1999]

Tamb�m definido em [AHO, 1995], um compilador � um programa que l� outro programa 
escrito em uma linguagem --- a linguagem de origem --- e o traduz em um programa 
equivalente em outra linguagem --- a linguagem de destino. Como uma importante 
parte no processo de tradu��o, o compilador reporta ao seu usu�rio a presen�a 
de erros no programa origem.

Ainda segundo [Rangel, 1999], existem duas tarefas triviais a serem executadas 
por um compilador nesse processo de tradu��o:

\begin{itemize}
 \item \textit{an�lise}, em que o texto de entrada (na linguagem fonte) � 
examinado, verificado e compreendido
 \item \textit{s�ntese}, ou \textit{gera��o de c�digo}, em que o texto de sa�da 
(na linguagem objeto) � gerado, de forma a corresponder ao texto de entrada.
\end{itemize}

Geralmente, pensamos nessas tarefas como fases que ocorram durante o processo de
 compila��o. No entanto, n�o se faz totalmente necess�rio que a an�lise de todo 
o programa seja realizada antes que o primeiro trecho de c�digo objeto seja gerado. 
Ou seja, estas duas fases podem ser intercaladas. Por exemplo, o compilador pode 
analisar cada comando do programa de entrada e ent�o gerar de imediato o c�digo 
de sa�da correspondente ao respectivo comando. Ou ainda, o compilador pode esperar 
pelo fim da an�lise de cada bloco de comando --- ou unidade de rotina 
(rotina, procedimentos, fun��es) --- para ent�o gerar o c�digo correspondente ao bloco.


