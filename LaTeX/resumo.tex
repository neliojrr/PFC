Usu�rios de computadores est�o cada vez mais conectados � internet. Necessitam ent�o que as ferramentas por eles
utilizadas estejam tamb�m dispostas de maneira online, para que possam ser acessadas de qualquer computador
ou dispositivo computacional.

Uma ferramenta bastante utilizada entre programadores e profissionais da �rea de computa��o � o compilador.
Mas para a cria��o de programas � sempre necess�rio que o computador utilizado tenha instalado ferramentas espec�ficas.
No entanto pode ocorrer de se usar um computador que n�o tenha tais ferramentas. � baseado nesse problema que este
trabalho prop�e a cria��o de um compilador e um sistema online de programa��o para a linguagem Portugol, visando
facilitar a programa��o, deixando-a independente de sistemas ou dispositivos computacionais.

Estudando as t�cnicas para a implementa��o de compiladores, bem como a cria��o de um sistema online, o prot�tipo
conseguiu obter bons resultados de desempenho e facilidade por parte do uso dos usu�rios. O prot�tipo chamado
PortugOn mostrou ser uma solu��o vi�vel para a programa��o atrav�s de navegadores.